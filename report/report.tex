\documentclass[12pt, a4paper,titlepage]{article}

\usepackage{hyperref}
\usepackage{fancyhdr}

\title{Development of a Component Based Share Trader Application}
\author{S. Dixon\\
        \href{mailto:40056761@live.napier.ac.uk}{40056761@live.napier.ac.uk}\\
        Edinburgh Napier University\\School Of Computing\\}
\date{April 2018}

\lhead{SET11504 - Advanced Software Development}
\rhead{40056761}

\begin{document}
\maketitle

\tableofcontents
\newpage

\section{Introduction}
The purpose of this report is to document and evaluate the
processes and techniques utilised in the construction of a component based
Share Trader desktop application.  The Share Trader system is a prototype
application that aims to aggregate and display information relating to shares,
trades, shareholders, brokers, etc. in a single, easy to use application.  The
caveat of developing such as system is that it must be constructed using
software components – software parts(?) that are reusable or modular in their
development and operation.  This report documents how software components were
identified, modified and combined to construct the Share Trader system – an
evaluation of the process and techniques is also provided.


\section{Component Mining}
When undertaking a Component Based Software Engineering (CBSE) project, the
process of identifying potentially usable software components is often
referred to as Component mining.  Component Mining consists of aggregating
units of existing software from multiple sources, giving a developer access to
to range of features that they can then quickly combine into a finished
system.

Components can be sourced from previous projects, open-source code
repositories, or stand-alone libraries. Having access to a wide range of
components can allow a prototype system to be quickly assembled and tested to
prove a concept.

\subsection{Legacy System}
The source files to a legacy system were provided from which potential
components could be identified and possibly incorporated into the Share Trader
system.  The legacy code was an implementation of  E-Store system and while
the business logic of the legacy system was not reusable – much of the
graphical user interface (GUI) had the potential to be reimplement as part of
the new Share Trader system.

The legacy system consisted of a number of GUI forms that were populated with
information stored in a database.

\subsubsection{Main menu}
The legacy system’s GUI consisted of a main menu featuring a number of buttons
– each which opened a new window to provide access to a different E-Store
feature; such as inventory management, delivery status, customer financial
status, etc.  The intuitive nature of the menu system made it an ideal
component to utilise in the new Share trader system – which would also consist
of a number features that a user would have to navigate between.

\subsubsection{Display forms}
When one of the legacy system’s menu buttons was clicked, a new GUI form would
open to display data and interactive elements to the user.  While the business
logic of these forms would be incompatible with the Share Trader system, the
tables used to display data, the text fields used for user input, and the
buttons utilised to submit searches and close the forms could be altered to
work with the new system.

\subsubsection{Database Connector}
Apart from the GUI elements, the legacy system also featured an interface
which defined the methods required to connect to a database. While the legacy
system’s concrete implementations would not be reusable – the database
connection interface could be used to implement a new database connection in
the Share Trader system.

\subsection{Previous Projects}
Database property system
A previous data analysis project completed by the the author (?) utilised
property files to augment and control access to a database, without the need
to alter or recompile source code. This component would allow the Share Trader
system to access different databases as required, or allow a developer to
connect to a one database for testing and another for deployment.

\subsection{Open-source libraries}
Software components are often published as stand-alone libraries, which can be
added to a project Class path and incorporated into the system. One such
library that was considered in this project was the JDatePicker – a GUI
component that creates a interactive calendar, allowing a user to select a
date. This component was considered as the project brief stated that users
would be able to search and sort trading information based on certain criteria
– date included. By utilising a pre existing date picker, development and
testing time could be saved.

\section{Component Selection}
From the previously mentioned components identified in the during the
component mining process, the following were utilised in the final system:
Database connection Interface
Database connection manager
JdatePicker


GUI layout and components

The justification for each is described;
The database connection interface not only provided sensible methods but also
allows a system to take advantage of the Object Orientated (OO) principal of
polymorphism – allowing a differing implementation of the interface to be
substituted for one another. In the case of the Share Trader system, the
database connection interface was implemented as a MySQL connection, but
implementations for any kind of data persistence system could be implemented
quickly and chosen at run time by utilising the existing database connection
interface.

The database connection manager was selected for use in this system primarily
for development reasons. The database management system allows users to
utilise a configuration (config) file to define the address, user, and
password needed to connect to a database. As the Share Trader system was
being developed across multiple devices each with a differing database
setup, the database management system allowed the system to function by
simply editing it’s config file, rather than its source-code.  The
Database management system also has the benefit of allowing a user to
define a primary and secondary database to which they can connect. This
can prove useful when testing a system, as it can connect to a database
set up as a testing environment, rather than performing potentially
devastating operations upon the data required for final deployment and
operation.

The JDatePicker is an Open-source component that allows a user to select a
date using a simple, clickable pop-up calendar. The JDatePicker has been
developed using Java Swing components – meaning it is compatible with the
other UI elements that were mined from the legacy systems codebase. The
JDatePicker's implementation has been completed using the MVC pattern – this
ensures the components compatibility with the rest of the Share Trader system
and allows for some slight customisation for it to better match the
specification of the system. One such alteration that the JDatePicker offers
is the ability to return selected dates in the SQL query format – ideal for
the proposed implementation of the Share Trader system.  Another benefit of
the JDatePicker is that the author of the component has thoroughly tested and
verified that the widget is working correctly. These verifiable tests reduce
the time required to test the new Share Trader system as there is no need to
retest a component that has already been tested(?).

The legacy system provided a substantial amount of inspiration of how the new
Share Trader system should be implemented. The simple and intuitive design of
the legacy system’s main menu can provide a foundation for the Share Trader
system – provided a few minor amendments are made to the labelled buttons.
Similar adjustments can be made to the forms used to display information
tables.  By reusing existing components from the legacy system, a substantial
amount of development time can be saved – utilising the work of others can
greatly reduce the pressure of development by reducing the amount of
responsibility on a single person or team.

\section{Component Adaptation}
In order to comply with the project specification, the  Share Trader system
had to be constructed with reusable components that were JavaBean compliant.
This meant that the potential components identified during Component Mining
process.

\subsection{JavaBeans}
A JavaBean is a class that complies to a standardised convention, meaning that
a system that is aware of said class’s parameters can automate the
instantiation of the class and it’s parameters. A JavaBean can also have its
current state serialised and saved – allowing a program to be paused entirely
and restored without hindrance.  In order for a class to be considered a
JavaBean, it must comply with the following:
Have a zero argument constructor
Provide property access via getters and setters
Be serializable

All the previously identified components, bar the JdatePicker, had their
source-code available, meaning that the requisite changes could be made to
ensure their compliance with the JavaBean standard.

\subsection{Model View Controller}
Taking further inspiration from the legacy system, and to ensure re-usability,
it was opted to develop the system using the Model View Controller (MVC)
design pattern.

MVC is a programming  pattern that separates the business logic (model) and UI
(view) of a system away from one an other, while a third controller element is
utilised to mediate and allow for communication between the model and view. 

MVC allows for UI and logic components to be developed individually from one
another as entirely separate reusable components.  The benefit of constructing
a program using the MVC  design pattern is that it allows for the different
MVC elements to be substituted for one another - for instance a text based UI
could be swapped for a graphical alternative, providing it implemented the
same required methods.
It also allows for the model elements to be altered should the business logic
of a system change, without any adjustments being made to the view or
controller subsystems.

\subsubsection{Model}
In an MVC designed system, models handle the logic operations of the program.
In order for a model to be reusable, it cannot rely on specific instances of
controller or view subsystems. In order to comply with the system
specification all of the models utilised in the Share Trader system where
implemented using the Observer design pattern –  making it so a subject
(model) can inform any observers (view or controller) without needing any
prior knowledge of how said observer operates.  By implementing each model
using Java’s inbuilt observable subclass, and having each model’s controller
linking it to a relevant view – it was possible to develop each element of the
Share Traders business logic atomicity – without any reliance on other models,
views or controllers.
By developing each model in this modular style it means that any aspect of the
business logic for the system can be reused in a future implementation or
rewritten without affecting any other part of the program.

\subsubsection{View}
Baring the main menu and login forms, the views to be used in the Share Trader
system were all similar in style – featuring a table to display relevant data
and a button to navigate back to the main menu. While some of the views
required additional components ; such as text input fields or additional
buttons, the basic table and button UI was used as a base, and the additional
components were added as needed.  JetBrains IntelliJ Idea features an inbuilt
Java Swing GUI creation tool that allowed the UI components from the previous
system to be quickly converted into MVC views, and additional UI components
(such as additional buttons or text fields) could be added as were needed.

All the views also implemented the Java Observer interface. This interface
allows the views to receive data from the  models, without the need of linking
them at compile time. This use of the Observer pattern allows a single view to
be attached to multiple models, for instance; the view used by users to log in
or register an account is attached to two models, one which handles all the
login operations and another that handles registering new users. The observer
and MVC patterns ensure that both the logic and UI elements of the system are
not reliant on one another and are thus reusable components.

\subsubsection{Controller}
Each view requires a separate controller class that is used to bind the
request model(s) to its view. In the Share Trader system the controller class
is used to instantiate, link the models to their view, and mediates this
connection by extracting data from the view when required and passing it to
the model. The MVC system makes means each view with its controller and model
is intendant from another – meaning these modules of logic and UI can be
swapped for replacements or used in a different application.  MVC allows for
dependencies to be injected at runtime, meaning that multiple views or models
can all be combined, and only the required system will be linked and usable
during deployment. This can be particularly useful if the system has to cater
to multiple tiers of users – such as regular customers and system
administrators in the case of the Share Trader application.
Each of the controllers in the Share Trader system was implemented as an
Observable class (much like the models). This was done so that a single,
global controller could observe each module, and pass or remove control from
them as required. The Global controller acts as a bridge between each of the
system modules, mediating and binding them into a coherent program.

The legacy systems Database connection interface required no adaptation to be
incorporated into the new Share Trader system – the Database management system
taken from a previous personal project did require some alterations however.
The database management system had been previously in an MVC designed project,
but it did not contain the same method definitions that were expressed by the
legacy system’s database connection interface. It was simple task to refactor
the Database Management systems method names and signatures to so that it
could be considered an instance of the Database connector.

As previously stated, the GUI forms from the Legacy system adapted into
stand-alone view components – that is to say that all system logic was removed
from the classes and the IntelliJ UI builder was used to fashion them into
stand-alone beans.  Additional the JTable class used to display data in a
tabular format was extended into its own custom component. This was done as
the Share Trader system would contain multiple forms used to display the
results of database queries – by converting the JTable to its own custom
component it removed the need to instantiate and link the multiple subclasses
needed to ensure the tables would behave correctly and consistently across the
system.

Finally the JDatePicker component required no adaption for it to integrate
with the new system. Due to it being developed with Java Swing components and
in the style of the MVC design pattern the JDatePicker could be attached to
any view via the IntelliJ Idea UI builder tool.

Due to fact the legacy system’s source code was available, it was possible to
make the required amendments directly to the class as required. This is a
preferable method of adaption, as other methods such as bridging and mediation
can often lead to unwanted features of a class being accessible or leading to
bugs and unspecified functionality.


\section{Component Integration}
Once components had been identified, adapted and converted into javaBeans and
MVC elements the construction of the system proper could begin.
 
In order to ensure the system comprised of reusable components, each section
or module of the Share Trader system was developed in isolation from the rest
of the system. This was done to ensure  each component had no reliance on any
other – forcing the MVC, component design paradigms.  Independent Git branches
were created for each system module (view, controller, and relevant models) to
ensure no accidental dependencies to other modules were created during the
development process.

Each system module was developed by using a similar process, starting with the
base view component that was mined from the legacy system. If the view
required additional components other than the default JTable and close button,
theses were added via the IntelliJ UI builder. An example of a view that
required additional editing was the view used to display trading history –
which featured a number of JTextFields a user could use to enter search
parameters.

Once a system module’s view was constructed, the requisite controller class
was built. All the controllers in the Share Trader system (barring the global
controller) extended the Observable class and implemented the ActionListener
interface. The reason controllers were made observable was so that any
information regarding the current focus of the Share Trader system could be
passed back to the global controller for processing.  The ActionListener
interface allowed the controller classes to process any user interactions with
the conjoined view. Whenever a user interacts with a UI component in a view –
such as a JButton or JTable – the controller will be informed of the
interaction and pass any required data either to the Global Controller or to a
model bundled as part of the same system module.
By having the controller modules implement the ActionListener interface, the
overall complexity of each system module was greatly reduced – as it was no
longer required to create a controller class for individual, interact-able
component of a view, in the same manner some traditional MVC patterns are
implemented.  The final element of each system module to be developed was the
model. Depending on the component, multiple MVC models were required to
implement all the required business logic – one example is the module used by
users to register accounts or login to the system, which featured two models
in order to deliver said functionality.
Each model in the share trader system extends the Observable interface. This
is required so that the model can pass any required data (such as MySQL query
results) to a view without a direct reference.  Other than extending the same
superclass, the individual models bared very little resemblance to one
another, as each implemented a unique piece of business logic required by the
Share Trader system.

Once the individual MVC elements for a component were constructed, they were
instantiated and linked via the modules controller class. The benefits of
developing a system via MVC have already been discussed extensively, but it is
worth highlighting an advantage of the design pattern is that if any of a
system module’s elements  required alterations, they could be made without the
need to modify any other classes.

A global controller class was created to instantiate and observe each
individual system modules. This global controller acts as mediator between the
other MVC modules, passing required data and showing and closing UI forms as
required.  As each MVC module implements a common interface, the global
controller consists of a simple switch to handle incoming and outgoing events
from the other modules.

\section{Testing}
Testing is crucial stage in the development of any system. Testing ensures the
the program being developed not only functions without error but also matches
its design specification. A number of testing and debugging techniques were
utilised to ensure the new Share Trader system was implemented correctly.

\subsection{Testing preparation}
In an ideal scenario, a system’s testing would be conducted by a third party
in order to prevent creators bias from affecting the quality or thoroughness
of test – this was not possible due to the nature of the project, thus all
tests were conducted by the author.

The following testing techniques were utilised to ensure system performance;
Unit testing – ensuring that an individual component or module function
correctly.
Integration testing – testing that multiple, joined components function
correctly.  Validation testing – ensuring the combined system meets all the
specified requirements.

\subsection{Unit testing}
Unit testing is where individual classes or modules of a system are tested and
examined to ensure correct functionality. As the Share Trader system was
developed in component, and MVC styles the process of unit testing came very
naturally to this project.  Once each element of a MVC module was completed, a
simple testing wrapper was developed  using the JUnit test framework to ensure
there was no interface mismatches or unintended features.

Depending on the class, an automated JUnit test could not always be written –
such as the JavaSwing UI components. Due to the nature of MVC no operational
code was contained within any of the UI classes, making it a trivial task to
use front-end testing and print-line debugging to ensure buttons and UI
components were making the correct system calls.

A second area of the system that was not suitable for automated unit testing
was the MySQL database queries. The Share Trader system mainly operates by
issuing queries to a database and displaying the results in an appropriate
format. Unfortunately, the process required to create automated tests for
MySQL commands in a Java environment is highly convoluted, requiring the
creation of a proxy database to receive and reply to queries – which was
considered outwit the scope of this project. In order to ensure that the MySQL
queries were correctly formatted each was written and tested in the MySQL
Workbench application. Once it was verified that each query was correctly
returning the expected results it was converted to a method in the appropriate
model class.

\subsection{Integration testing}
Once a component has been thoroughly unit tested, it can be linked to other
system components or modules and integration testing can begin. Integration
testing is the process of ensuring linked systems or components are behaving
as expected. If component's interfaces and methods have been well designed and
thought unit tests has taken place there should be a minimal chance for
unintended behave to occur once systems are linked – nonetheless it is crucial
to perform integration system on every software system.  For the Share Trader
system, integration testing was performed by adding unit tested modules to the
global controller class and ensuring the correct system calls were made when
each subsystem was instantiated, shown, and closed.
Due to the unit testing performed previously, the integration testing revealed
very few bugs or unintended features in the system. This can be credited to
the effort spent during the component creation stage of the project – leading
to class of a simple design, with little chance for unintended function.

\subsection{Validation testing}
The final phase of testing is validation testing – ensuring that the system
complies with its design specification.  For this project, no formal design
specification was provided and all business
logic and use cases for the Share Trader system were defined by the developer.
In a real-world scenario, system specifications would be constructed by a
third party or a client – much like the ideal case for unit tests.  Validation
testing was conducted by simply running the Share Trader system and executing
the tasks outlined by the project specification.
During the process of validation testing, if it discovered that a component or
module did not properly fulfil its specification, amendments could be made.
As has been reinstated multiple times – the MVC and component based design of
the Share Trader system allowed alterations to be made with minimal affect on
other components.

\section{Evaluation}
It is crucial to evaluate not only the quality system, but the methods and
techniques utilised in it construction.

\subsection{System Evaluation}
The completed Share Trader system implemented all the requested features, sans
the news alert system. The news alert system was omitted from the prototype
due to time constraints, and also because the prototype system already
features an alert system to inform users of when a tracked share is altered –
displaying the same functionality, in a slightly different format.

In terms of system’s appearance, the default JavaSwing components were
utilised, leading to a simple but utilitarian UI. It could be argued that the
systems appearance was dull or inspired, but as the goal of this project was
to produce a proof of concept of a component based Share Trader system – and
as such the appearance of the system is outwit the scope of this project.

There does exist a security flaw system, as the text field utilised by users
to enter search criteria are not satanised. This flaw means it would be
possible for a user to make alterations to, or even delete the data stored in
the system’s database. The risks related to this flaw could be mitigated by
utilising Java SQL’s PreparedStatement class to satanise queries that include
any form of user input. The use of PreparedStatments was omitted to reduce
development time – regular Java SQL statements were utilised so that multiple
queries could be combined, and amendments could be made during the development
and testing phases of the project.  While the project specification makes to
reference any security requirements, it has been noted that the inclusion of
unsanitized inputs is a poor software engineer practice and should be avoided
in real-world projects – an exception was only made for the Share Trader
system as it is a prototype proof of concept.

Overall, the share trader prototype system fulfils its design and testing
specification and provides the functionality requested. While the news alert
system was committed, identical functionality is provided elsewhere in the
project – which could be reworked to deal with news alerts as well.


\subsection{Method Evaluation}
The development process of the share trader system utilised a number of
software engineering methods and techniques to ensure a stable and error free
component  based system was produced.  

\subsubsection{Component Mining}
Reusing pre-written source-code or existing libraries when building a software
system can significantly reduce the time required both during the development
and implementation stages, as well as the testing phases of the project.

The specification for this project stated that it must contain components
taken from the legacy system’s codebase. This step was somewhat unnecessary
for a project of this scale and simplicity, as time used scouring the legacy
codebase for usable   and scalable code snippets might have been better
utilised in other areas of this project.

One of the issues caused by forced code reuse is that it can obligate the use
of a particular technology or library that may not be suitable for the
project. An example of this would be the reuse of the legacy system’s UI
elements forcing the use of Java Swing, a tool which has become steadily less
popular due to the rise of newer, more feature rich UI tools such as Spring or
JavaFX. While Swing is not entirely unsuitable choice for the production of a
prototype system, it does come with a substantial overhead in terms of
boilerplate code.

It cannot be said the code reuse does not have its benefits, an example of
this in the Share Trader system is the identification and reuse of the
Database Connection interface from the Legacy system’s source code. The
implementation and reuse of interfaces is not only a simple task, but can also
lead to backward compatible code and code constituency across multiple
systems.

\subsubsection{Component Adaptation}
For the Share Trader system the adaptation of components was a simple process,
due to having direct access to components source-code. It can often be the
case when building a system from existing components that an interface
mismatch can occur. Interface mismatch is when different components method
input parameters and output types do not match or are incompatible, meaning
that they cannot be properly linked or communicate.  Fortunately the
components mined from the legacy system were available as source code –
meaning that interface mismatches could be easily rectified  using Generative
adaption rather than more convoluted bridge or wrapper approaches.

Editing components source code may not technically classify as adaptation, but
it is a simple process that allowed a significant amount of the legacy systems
source to be retained, while also removing and reworking unneeded or
unsuitable portions of code. The process of adapting existing components can
not only save development and testing time, but also provide inspiration on
how to implement other parts of the system – for instance; the MVC design of
the Share Trader system was inspired from the partial MVC implementation in
the legacy System.

\subsubsection{Component integration}
There is very little to be discussed in regards of the process used to
integrate the components used in this project. The time spent during the
component mining and adaptation phases of the project minimised the time
required to integrate the various components into a full system and any errors
or bugs were able to be quickly located and amended thanks to the modular
design of the components and their MVC implementation.

\section{Conclusion}
When utilising component based software engineering methods, it is crucial to
account for the scale of the project. For the Share Trader system documented
in this report, the use of CBSE methods may have been somewhat overblown when
the overall size and complexity of the system is considered.  It cannot be
said that CBSE methods are not useful, nor that they do not have place in
software production – enterprise level systems, developed by organisations
with existing codebases posses the economies of scale that lead to desining,
implementing, and testing in a CBSE style saving them a significant amount of
time, and in return financing. 

\end{document}

